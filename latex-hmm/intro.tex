%!TEX root=main-hmm.tex

%-------------------------------------------------------------------------------
% Introduction
%-------------------------------------------------------------------------------
\section{Introduction}
\label{sect:intro}

% Motivation:

% The geo-location data published on-line has two advantages over the traditional
% GPS data collected by attaching GPS chips to monitor the target object. First,
% the geo-location data usually has semantic descriptions. For example, a Twitter user
% ususally composes a tweet to describe the activity or the visited place. Second,
% the amount of such geo-location data is far larger than manually collected GPS data.
% For instance, there are more than 500 million geo-tagged tweets published every day.


% The geo-location data published on-line are mostly sparse.
% Not all the transitions are reliable, we need to downweight the spurious or less
% relevant transitions.

% For frequent sequential pattern mining, a small support threshod
% generates too many patterns, which may increase the complexity for 
% people to understand the data, rather than reducing it.

% We want to find the inherent and interpretable mobility patterns from large-scale
% geo-location data. Note that the challenges are as follows. First, there are usually 
% a lot of untrustworthy movements in the geo-location data. Second, we need to simultaneously
% consider the geographical and textual descriptions of the user checkins.

% We design a unified model that simultaneously detects outlier movements and extracts
% interesting mobility patterns.
